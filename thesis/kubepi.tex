\documentclass[12pt]{report}
\usepackage[T1]{fontenc}
\usepackage[utf8]{inputenc}
\usepackage{graphicx}
% \usepackage{wrapfig}
\usepackage{amsmath,amssymb,amsfonts}
\usepackage{txfonts}
% \usepackage{placeins}
\usepackage{listings}
\usepackage{pdfpages}
\usepackage{color}
% \usepackage{framed}
% \usepackage{filecontents}
\usepackage{caption}

\usepackage[polish]{babel}

\renewcommand{\chaptername}{Rozdział}
\renewcommand{\contentsname}{Spis treści}
\renewcommand{\figurename}{Rys.}
\renewcommand{\tablename}{Tab.}
\renewcommand{\listfigurename}{Spis rysunków}
\renewcommand{\listtablename}{Spis tabel}
\renewcommand{\bibname}{Bibliografia}

\pagestyle{headings}

\setlength{\textwidth}{14cm}
\setlength{\textheight}{20cm}

\begin{document}

\tableofcontents    % generuje spis treści ze stronami !!!

\chapter{Wstęp}\label{chap:wstep}

\section{Problematyka i zakres pracy}
\section{Metoda badawcza}
\section{Przegląd literatury w dziedzinie}
\section{Układ pracy}

\chapter{KubePi}\label{chap:kubepi}
\chaptermark{KubePi}

\section{Podstawowe definicje}
\section{Istniejące aplikacje}
\section{Wymagania aplikacji}
\section{Wady i zalety istniejących rozwiązań}

\chapter{Projekt KubePi} \label{rozdz.czesc.prakt}
\section{Analiza wymagań}
\section{Użyte technologie}
\section{Projekt}
\section{Opis użytkowania}
\section{Przykład użycia}
\section{Możliwości rozszerzania aplikacji}
\chapter{Podsumowanie}
\section{Dyskusja wyników}

\addcontentsline{toc}{chapter}{Bibliografia}
\begin{thebibliography}{99}
	\bibitem{go} {Some books}
\end{thebibliography}

\addcontentsline{toc}{chapter}{Spis rysunków}
\listoffigures

\addcontentsline{toc}{chapter}{Spis tabel}
\listoftables

\end{document}
