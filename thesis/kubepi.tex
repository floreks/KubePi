\documentclass[12pt]{report}
\usepackage[T1]{fontenc}
\usepackage[utf8]{inputenc}
\usepackage{graphicx}
\usepackage{wrapfig}
\usepackage{amsmath,amssymb,amsfonts}
\usepackage{txfonts}
\usepackage{placeins}
\usepackage{listings}
\usepackage{pdfpages}
\usepackage{color}
\usepackage{framed}
\usepackage{filecontents}
\usepackage{caption}

\usepackage[polish]{babel}

\renewcommand{\chaptername}{Rozdział}
\renewcommand{\contentsname}{Spis treści}
\renewcommand{\figurename}{Rys.}
\renewcommand{\tablename}{Tab.}
\renewcommand{\listfigurename}{Spis rysunków}
\renewcommand{\listtablename}{Spis tabel}
\renewcommand{\bibname}{Bibliografia}

\pagestyle{headings}

\setlength{\textwidth}{14cm}
\setlength{\textheight}{20cm}

\begin{document}

\includepdf{images/title-page.png}

\tableofcontents    % generuje spis treści ze stronami

\chapter{Wstęp}\label{chap:wstep}
\section{Problematyka i zakres pracy}
Wraz z rozwojem Internetu Rzeczy na świecie powstają coraz to nowe urządzenia mające na celu automatyzację działań i ułatwienie życia człowieka. Dzięki ich zdolności do wzajemnej komunikacji, wymiany informacji oraz zdalnego zarządzania zasobami stają się one coraz bardziej popularne, a wręcz wymagane w życiu codziennym coraz większej grupy osób. Sterowanie oświetleniem, radiem czy innymi sprzętami elektronicznymi za pomocą naszego smartfona już nikogo nie dziwi. W wielu przypadkach urządzenia te muszą zbierać bardzo duże ilości danych oraz przesyłać je do centralnego punktu. Powoduje to ogromny wzrost ilości danych, które nie są w stanie zostać obsłużone przez jeden serwer. Powstaje więc potrzeba stworzenia niezawodnych, wydajnych i bezpiecznych systemów o wysokiej dostępności. \\
\indent Technologie wirtualizacji
\footnote{\cite{virtualization}}
powstały w celu realizacji tych wymagań. Wirtualizacja serwerów dawno już wyparła tradycyjne serwery, które zostały zastąpione przez rozwiązania chmurowe. Niezależność sprzętowa, lepsza utylizacja zasobów, większe bezpieczeństwo, łatwa migracja danych i redukcja kosztów to tylko niektóre z wielu zalet wirtualizacji. Właśnie ta niezależność sprzętowa pozwala na coraz lepsze wykorzystanie urządzeń opartych na architekturze ARM, których głównymi zaletami są mały koszt i niewielki pobór mocy, a dzięki coraz lepszej optymalizacji systemów i postępującej miniaturyzacji, również rosnąca wydajność. \\
	\indent Proponowanym rozwiązaniem powyższych problemów będzie projekt o nazwie KubePi. Projekt ten skupia się na wirtualizacji opartej o kontenery Dockera
	\footnote{\cite{docker}}
zarządzane przez system zarządzania klastrem Kubernetes i ma na celu stworzenie rozproszonego systemu służącego do monitorowania otoczenia. Przykładowy klaster opierać się będzie na dwóch mikrokontrolerach RaspberryPi. Do pierwszego urządzenia będącego zarazem głównym węzłem klastra podłączone zostaną trzy czujniki: temperatury, wilgotności oraz alkoholu. Jego zadaniem będzie udostępnianie zbieranych informacji. Drugie urządzenie zostanie natomiast wyposażone w wyświetlacz LED
	\footnote{\cite{led}}
, co pozwoli na odczyt i wyświetlanie temperatury raportowanej przez pierwsze urządzenie. Dodatkowo w klastrze zostanie zainstalowana aplikacja webowa
\footnote{\cite{webapp}}
pozwalająca na zdalny monitoring. W celu lepszego zobrazowania komunikacji między urządzeniami zostanie ona uruchomiona na drugim urządzeniu.

\section{Metoda badawcza}
% TODO

\section{Przegląd literatury w dziedzinie}
% TODO

\section{Układ pracy}
Celem pracy jest zaproponowanie architektury i sprawdzenie w działaniu rozproszonego systemu wysokiej dostępności służącego do monitorowania otoczenia. \\
\indent Rozdział pierwszy zawiera szczegółowy opis problemu. Zostają w nim przedstawione różne problemy związane z wydajnością oraz bezpieczeństwem tradycyjnych rozwiązań, wraz z opisem metod badawczych użytych do analizy tematu. Podsumowane zostają również główne założenia i cele pracy. Na koniec przeprowadzony zostaje przegląd literatury związanej z tematem, z naciskiem na kluczowe zagadnienia dotyczące wirtualizacji, rozwiązań chmurowych oraz mikrokontrolerów opartych na architekturze ARM, wraz z krótkim opisem użytych źródeł. \\
\indent W rozdziale drugim przybliżona zostaje tematyka systemów zarządzania klastrami pod kątem ich wymagań, bezpieczeństwa oraz komunikacji sieciowej. Kolejnym krokiem jest dokładniejsze zapoznanie się z wirtualizacją, a konkretniej wirtualizacją opartą o kontenery Dockera, co pozwoli lepiej zrozumieć ideę pracy. Następnie po krótce przedstawione zostają tematyki związane z mikrokontrolerami oraz Internetem Rzeczy. \\
\indent Rozdział 3 skupia się na analizie istniejących systemów zarządzania klastrami oraz ich pochodnych. Dodatkowo przedstawiona zostaje analiza kilku systemów typu Smart Home
\footnote{\cite{smarthome}}
. \\
\indent Kolejny rozdział opisuję fazę projektowania i implementacji projektu KubePi. Spisane zostają wymagania funkcjonalne aplikacji, a także ograniczenia projektowe. Wymienione i opisane zostają użyte technologie. Opisany zostaje proces konfiguracji urządzeń, sieci oraz systemu. Następnie wskazane zostają kluczowe punkty aplikacji wraz z kodem źródłowym i opisem. W kolejny kroku przechodzimy do fazy testów stworzonych aplikacji jak i całego systemu. \\
\indent W podsumowaniu pracy opisane zostają słabe i mocne strony przedstawionego rozwiązania. Na podstawie uzyskanych wyników następuje ocena możliwości i przydatności zaproponowanego rozwiązania. Na końcu omówione zostają możliwe perspektywy rozwoju projektu.

\chapter{Podstawy teoretyczne}\label{chap:background}

\section{Podstawowe definicje}
\subsection{Wirtualizacja oparta na Linuksie}
\subsection{Wirtualizacja oparta o nadzorcę}
\subsection{Wirtualizacja oparta na kontenerach}
\subsubsection{Docker}
\subsection{Mikrokontroler RaspberryPi}
\subsubsection{Protokoły komunikacji}
\subsection{Kompilacja skrośna}

\section{Systemy zarządzania kontenerami}
\subsection{Kubernetes}
\subsubsection{Architektura}
\subsection{Docker Swarm}
\subsection{Resin.io}

\chapter{Projekt KubePi} \label{rozdz.czesc.prakt}
\section{Analiza wymagań}
\section{Użyte technologie}
\section{Projekt}
\section{Opis użytkowania}
\section{Przykład użycia}
\section{Możliwości rozszerzania aplikacji}
\chapter{Podsumowanie}

\addcontentsline{toc}{chapter}{Bibliografia}
\begin{thebibliography}{99}
	\bibitem{go} {Some books}
\end{thebibliography}

\addcontentsline{toc}{chapter}{Spis rysunków}
\listoffigures

\addcontentsline{toc}{chapter}{Spis tabel}
\listoftables

\end{document}
